Crowdsourcing media annotation approaches are used in various applications and are used to gather information of various types, such as temporal synchronization\cite{wu2014crowdsourced}, events\cite{Kim:2014:JSL:2679600.2680027}, scene objects\cite{vidwiki2014}, emotions\cite{biel2013youtube}, actions\cite{riek2011guess}, and geo-tagging\cite{gottlieb2012pushing}. Among these works, \cite{Kim:2014:JSL:2679600.2680027,riek2011guess,wu2014crowdsourced} used microtasking approaches and unskilled workers, while \cite{vidwiki2014,biel2013youtube,gottlieb2012pushing} required skilled or trained workers to perform more complex tasks.

The work conducted by Kim \cite{Kim:2014:JSL:2679600.2680027}, he used a crowd to jointly summarizing large sets of Flickr\footnote{https://www.flickr.com} images and YouTube\footnote{https://youtube.com} videos in order to create novel structural summaries of online images as storyline graphs. The workers received a set of frames extracted from a video segment, and had to select some of them to create a summary. Once the set of convergent images were obtained, they were used to find similar images on a dataset extracted from Flickr.

Riek used a game approach \cite{riek2011guess} to annotate a video dataset with tags related to facial expressions, posture, gestures and more. The crowd used a very simple annotation tool in which they could insert tags by clicking buttons.

VidWiki\cite{vidwiki2014} is a complex system to improve video lesson by annotating them, which provides a complex annotation tool that allows the worker to edit video scenes by adding various types of annotations, including LaTex equations. This system requires some skills, including knowledge about Latex\footnote{https://www.latex-project.org}.

An important observation of the works related to crowdsourcing media annotation is that, in general, microtasks are performed by unskilled workers using simple annotation tools, to obtain simple annotations. On the other hand, jobs that aim at complex annotation often use larger tasks that require skilled or trained workers as well as more elaborate tools.