This paper introduced a crowdsourcing approach to annotate videos without requiring experts, trained workers or time-consuming tasks. Moreover, was conducted an experiment to validate it by generating interesting annotated videos that could be used to create interactive multimedia presentations. To support this experiment was developed a toolkit that includes the presentation system, and a set of video annotation tools and aggregation methods.

During the annotation stage, it was noticed that the faster microtasks received more contributions, because the workers contributed more times, annotating more items. One conclusion about this is that volunteers use to dedicate a set time to perform tasks, so they were willing to execute any number of microtasks during that interval.

Another observation about the approach is that the cascade of tasks results in the generation of partial results that can be used for other purposes. For example, content suggestions that have been collected to annotate the video can be used to populate an online dictionary or encyclopedia.

Moreover, the individual aggregation of the result of each microtask allows an adequate processing for each annotation, as well as specific validations for them.

Perhaps one of the most interesting results was to see if this approach is capable of generating systems that can be reused and expanded. This can be observed when the first presentation system was generated and later a new task was added in the process, allowing the construction of an improved presentation system.

In addition to the approach presented, which was able to guide crowdsourcing annotation processes with a certain degree of complexity, a system was also generated that demonstrates how this approach can be applied. This system is available for use and can be used both to replicate this experience and to perform other works.



\subsection{Next Steps}

An immediate improvement in the system includes changes in the aggregation methods of tasks 1 and 2. Currently, the similarity comparison uses simple syntactic techniques for content analysis. However, a method is being developed that performs these comparisons through morphosyntactic analysis.

The owner module will also be developed, which will allow this system to be used even outside the academic environment. Currently, the system counts only as microtask execution module, which was necessary to perform the experiment.

This work also served as a starting point for a series of projects that will be developed shortly. In particular, the approach presented will be refined to become a complete method.
