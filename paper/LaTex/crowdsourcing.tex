%Human computation approaches can improve performance by the division of labor because it helps to execute tasks in parallel. Each worker performs their work independently so that the instances of a task can be executed in parallel, according to the Human Computation paradigm \cite{Rohwer:2010:NHC:1837885.1837897}.

To support the human computation paradigm, crowdsourcing has emerged as a proposal to annotate media objects using a large number of contributors efficiently \cite{VonAhn:2005:HC:1168246}. 
%Following the crowdsourcing principles, the tasks distributed to the workers are modeled to be done independently, maximizing the parallelism \citep{Howe2006}. Furthermore, each task can be sent to many contributors, allowing to compare, check and to aggregate the contributions, also reducing the chance of producing a biased result  \cite{GALTON1907}.

This approach generally delivers good quality results using contributions from the crowd and can distribute, collect, validate and combine large amounts of tasks results \cite{Mo:2013:OPH:2505515.2505755}. Since this approach is designed to handle a huge number of cooperators and contributions for tasks that require human intelligence \cite{Howe2006}, Crowdsourcing is appropriate to allow the Human Computing paradigm to be applied in a massive-scale online cooperation \cite{TEDMassive}. Crowdsourcing  is supported by four pillars:  The Crowdsourced Task, The Crowdsourcer, The Crowd, and The Crowdsourcing Plataform \cite{6861072}.

\begin{itemize}

\item{\textbf{The Crowdsourced Task}} is the HIT designed, according to the Human Computing paradigm, to acquire workers' contributions. Task instances are presented to the workers as jobs that must be performed \cite{Difallah:2015:DMC:2736277.2741685}.

\item{\textbf{The Crowdsourcer}} is the owner of a project, it may be an individual or institution that wishes to have a completed task. The owner is responsible for starting the crowdsourcing process, defining what task must be completed and how it should be presented to the workers as jobs \cite{6861072}.

\item{\textbf{The Crowd}} is the workforce that moves the process once it is composed of all the workers who perform the jobs needed to generate the outcome. Each worker carries out his work independently so that the works can be executed in parallel, according to the paradigm of Human Computation \cite{Rohwer:2010:NHC:1837885.1837897}.

\item{\textbf{The Crowdsourcing Platform}} is a computational system responsible to manage the whole process, serves as an entry point for both the owner making the tasks available and for the workers to execute them. This kind of environment can be something sophisticated such as CrowdFlower, Microworkers, and AmazonMechanical Turk \cite{Difallah:2015:DMC:2736277.2741685}, or really simple systems with screens and forms for data collection such as the mobile application used by the Google Crowdsource project \cite{google_cs}. A crowdsourcing environment is necessary, as the tasks must be made available to a potentially large number of workers. The crowdsourcing platform is a key element of support to massive-scale cooperation.

\end{itemize}

The use of a commercial crowdsourcing platform brings benefits such as not having to worry about management's issues of workers, jobs, and contributions such as employee recruitment and collection of contributions as well as facilitating employee payments. Also, payouts are a good way to motivate and keep the crowd, although there are other coping factors such as personal accomplishment and gamification.



